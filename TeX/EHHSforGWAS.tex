\documentclass[a4paper,11pt]{article}
\usepackage[latin1]{inputenc}
\usepackage{times}
\usepackage[T1]{fontenc}
\usepackage[english]{babel}
\usepackage{latexsym}
\usepackage{tipa}
\usepackage{pictex}
\usepackage[authoryear,round]{natbib}
\setcounter{secnumdepth}{0} \linespread{1.5}
 
\usepackage{amsmath}
\usepackage{amssymb}
\usepackage[usenames,dvipsnames]{color}
\usepackage{graphicx}
\usepackage{verbatim}
\usepackage{amsmath}
\usepackage{amsfonts}
\usepackage{amssymb}
\usepackage{latexsym}
\usepackage{natbib}
\usepackage{xr}
\usepackage{setspace}
\usepackage{framed}
\usepackage{cancel}

\newcommand{\comb}[2]{{#1 \choose #2}}
\newcommand{\mosr}[1]{{ #1}}

\begin{document}

\title{Detection of the effect of selected phenotypic traits using additional statistics in GWAS.}
%\author{Sebastian E. Ramos-Onsins, Ioanna Theoni Vourlakis, \\ Josep Ma. Folch, Miguel P\'erez-Enciso\\ (Contributed: not yet the real order)}
\date{\today}
\maketitle

We propose to use the quantity of linkage disequilibrium associated to a given position as a ponderation for each of the individuals at each given studied position. This statistic will be used in GWAS analysis of association to detect the effect of adaptive selection of the studied phenotype. If we consider a single population that have suffered a strong selective event, those causal positions that strongly determined the selective effect will increase rapidly their frequency, and thus will have larger linkage disequilibrium with the closest sequences than other regions not affected by selective events.  With this idea it is pretended to take advantage of the stronger footprint of a sweep associated to a selective event  to give more weight to those individuals in whose the studied position contains a determinant allele, as well as longer linkage in close regions caused by the effect of the selective sweep. Our hypothesis is that this approach can have  promising results to detect the causal mutations in panmictic populations that have experimented stronger selective events. On the other hand, we think that can not give additional information in GWAS studies designed to distinguish between divergent populations (or control-disease populations). A consistent simulation study is required in this project to evaluate different scenarios.

The GWAS analysis with a single panmictic population is based on the association analysis of a single variant position for each of the individuals (usually genotype information, that is, a variable with  the discrete values 0, 1 and 2, indicating the number of copies of a variant - reference or alternative -), considering the relationship between individuals and other variables, versus the observed phenotype at each of the individuals. Thus, it is crucial to define all the genetic features at the studied position for each individual. One of this genetic features is the degree of homozygosity/heterozygosity in neighbouring regions to this position at each of the individuals. The presence of causal variants close to a given position and the observation of homogeneity of the individual at this region will, presumably, be associated with the phenotype. The larger homogeneity on the focus position with their neighbouring region may indicate a causal effect.

The resolution of this kind of statistics is expected to be in the range of the domestication process (not too short, approximately from several hundreds of generations to  thousands of generations, REFERENCE). This range may allow to detect critical traits and the effect of these traits on the genome under crucial events such as the domestication.

\section{Methods}
\subsection{Considering a single population and polarized alleles (ancestral and derived)}
Following \cite{Sabeti:2002fv}, we define the statistic $EHH_d$ as the proportion of homozygote individuals of the derived allele from position $i$ (the target position) to position $j$, in relation to the number of homozygote derived individuals at the position $i$.  That is:

 \begin{equation}
 EHH_{d_{ij}} = \frac{\sum_{k=1}^{n}I_{d_{k,ij}}}{\sum_{l_d=1}^{n}I_{d_{l,i}}},
 \end{equation}
where $n$ is the number of individual samples, and $I_{d_{k,ij}}$ counts 1 if the individual $k$ is homozygote for derived variant at position $i$ and is homozygote from position $i$ to $j$, otherwise counts 0. In the same way, $I_{d_{l,i}}$ counts 1 if the individual $l$ is homozygote for derived at position $i$, otherwise counts 0. The statistic $EHH_a$ is calculated in the same way than $EHH_d$ but considering ancestral variant instead of derived variant.  

In order to quantify the effect of the neighbouring homozygosity at a given studied position ($i$),  all values for $EHH_{ij}$ are summed from  the position $i$,to their right and and their left sides, all the contributions of $EHH_{ij}$, considering their distance (physical or recombinant), and until a threshold arbitrary value of $EHH_{ij}$ (let's say 0.1). That is: 

\begin{equation}
 iES_{d_i} = \sum_{j=x+1}^{y}\frac{(EHH_{d_{ij-1}} + EHH_{d_{ij})}}{2} (Pos_{j} - Pos_{j-1}),
 \end{equation}
where $x$ and $y$ are the positions (at the right and at the left) where $EHHS_{d_{ij}}$ becomes bellow the threshold or it is too far (by the presence of large gaps) from the central position (so the area out of $x$ and $y$ is considered unimportant), where $Pos$ may be the physical or the recombinant position (from a linkage map). The same is for the ancestral statistic ($iES_{a_i}$).

The integrated relative Extended homozygosity statistic is then calculated as:
 \begin{equation}
 iRES_{da_{i}} = \frac{iES_{d_{i}}} {iES_{a_{i}}}.
 \end{equation}
A value higher than 1 (if the genotype extension statistic for derived variants is larger than for ancestral variants) suggest the effect of positive selection favouring a derived variant at position $i$.  

\subsubsection{Quantifying homogeneity at each position and individual}
In case of doing a GWAS analysis, usually each position is evaluated independently in relation to the phenotype. That means that it is crucial to distinguish the genetic information between individuals at this position. Here we propose to include the information concerning to the homogeneity of the individual at neighbouring regions from the focus position (that is, a way to consider the linkage disequilibrium between the two chromosome copies of an individual at this region) in addition to the genotype at the focus position. That is, it is calculated a new statistic related to the neighbouring homogeneity for every position and individual. This statistic is obtained by slicing the $iRES_{da_i}$ statistic given the contribution of each individual to the total:

 \begin{equation}
 iRES_{da_{k,i}} =  \frac{iES_{d,k}}{iES_{a,k}} = \frac{\sum_{j=x+1}^{y}(EHH_{d_{k,ij-1}}+ EHH_{d_{k,ij}})(Pos_{j} - Pos_{j-1})}  {\sum_{j=x+1}^{y}(EHH_{a_{k,ij-1}}+ EHH_{a_{k,ij}})(Pos_{j} - Pos_{j-1})},
 \end{equation}
where $x$ and $y$ are the positions (at the right and at the left boundaries) and $EHH_{d_{k,ij}} = I_{d_{k,ij}} / (\sum_{l_d=1}^{n}I_{d_{l,i}})$. Note that $\sum_{k=1}^{n}iES_{d,k}=iES_d$ but the relationship between $iRES_{da_{i}}$ and $iRES_{da_{k,i}}$ is not the sum of all $iRES_{da_i}$ because of the quotient.

\subsection{Contrasting a population against a reference population}
Here, we will use the framework developed by \cite{Tang:2007by} to study the degree of homogenization of a given position (using unphased genotype information).
\subsubsection{Define candidate positions}
 In a first step, it is useful to define those candidate positions in relation to their maximum local extension of their homozygosity. This step is optional, as other criteria for defining candidate positions can be used. Following \cite{Tang:2007by}, we define $EHHS_{ij}$ as the proportion of homozygote individuals from position $i$ (the position of interest) to position $j$, in relation to the number of homozygote individuals at the position $i$. That is:
 
 \begin{equation}
 EHHS_{ij} = \frac{\sum_{k=1}^{n}I_{k,ij}}{\sum_{l=1}^{n}I_{l,i}},
 \end{equation}
where $n$ is the number of individual samples, and $I_{k,ij}$ counts 1 if the individual $k$ is homozygote from position $i$ to $j$ (\textit{i.e.}, using the genotype nomenclature, all variants have the values 0 or 2 at this region), otherwise counts 0. In the same way, $I_{l,i}$ counts 1 if the individual $l$ is homozygote at position $i$, otherwise counts 0. This statistic is calculated from position i to any position (left or right) until this proportion becomes enough small to be considered negligible. This threshold value, although is somewhat arbitrary, it has been considered 0.1 in the original work \citep{Tang:2007by} and here it is used the same criteria. The $EHHS_{ij}$ values for the position $i$ are kept and used to calculate the next statistic $iES_i$ \citep{Tang:2007by}.

The following calculation of the $iES_i$ statistic pretends to quantify the effect of the neighbouring homozygosity at a given studied position. Having all values for $EHHS_{ij}$, we count the total area of homozigosity around the position $i$, that is, having the position $i$ as the center, we sum, at their right and and their left, all the contributions of $EHHS_{ij}$, considering their distance (physical or recombinant). that is: 

 \begin{equation}
 iES_i = \sum_{j=x+1}^{y}\frac{(EHHS_{ij-1} + EHHS_{ij})}{2} (Pos_{j} - Pos_{j-1}),
 \end{equation}
where $x$ and $y$ are the positions (at the right and at the left) where $EHHS_{ij}$ becomes bellow the threshold or it is too far (by the presence of large gaps) from the central position (so the area out of $x$ and $y$ is considered unimportant), where $Pos$ may be the physical or the recombinant position (from a linkage map).
 
\subsubsection{Quantifying homogeneity at each position and individual}
In case of doing a GWAS analysis, usually each position is evaluated independently in relation to the phenotype. That means that it is crucial to distinguish the genetic information between individuals at this position. Here we propose to include the information concerning to the homogeneity of the individual at neighbouring regions from the focus position (that is, a way to consider the linkage disequilibrium between the two chromosome copies of an individual at this region) in addition to the genotype at the focus position. That is, it is calculated a new statistic related to the neighbouring homogeneity for every position and individual. This statistic is obtained by dividing the $iES_i$ statistic given the contribution of each individual to the total:

 \begin{equation}
 iES_{k,i} = \frac{1}{2\sum_{l=1}^{n}I_{l,i}} \sum_{j=a+1}^{b} (Pos_{j} - Pos_{j-1}) (I_{k,ij-1} + I_{k,ij}),
 \end{equation}
where $\sum_{k=1}^{k=n}iES_{k,i}=iES_i$ (considering the same threshold values -$a$ and $b$- for the each of the samples, lke $iES_i$ statistic). Here the only differential term between individuals (in relative terms) is the last sum. % Note also that the term $2\sum_{l=1}^{n}I_{l,i}$ can be eliminated from the equation because %
%That is, the objective here is the difference between individuals and not the absolute value. Therefore:

%Additionally, if we want to count not only the homogeneity in homozygous positions but also in heterozygous positions, we can define $I'_{k,ij}$, where it counts 1 if the genotype between $i$ and $j$ are completely homozygous or completely heterozygous.  %
%Then the new statistic $iES'_{k,i}$ will consider also heterozygous regions (for example, in case having advantage for heterozygotes). That is:

%\begin{equation}
% iES'_{k,i} = %
%%% \frac{1}{2\sum_{r=1}^{n}I_{r}} \sum_{j=a+1}^{b} (Pos_{j} - Pos_{j-1}) (I'_{k,ij-1} + I'_{k,ij}) \propto 
%%{\sum_{j=a+1}^{b} (Pos_{j} - Pos_{j-1}) (I'_{k,ij-1} + I'_{k,ij})}.
%(\frac{1}{2\sum_{l=1}^{n}I_{l,i}} \sum_{j=a+1}^{b} (Pos_{j} - Pos_{j-1}) (I_{k,ij-1} + I_{k,ij}))/iES_i,
%\end{equation}
%and $\sum_{k=1}^{k=n}iES'_{k,i}=1$.
%%The limit of the positions ($a$ and $b$) can be set to those positions where the sum of all $I'_{k,ij}$ in relation to the number of individuals drops below 0.1, as the threshold established for the statistic $iES_i$. 

\subsubsection{The quotient between the extension of homozygosity in target individuals from a population versus a reference population}
Following the same reasoning, it is possible to estimate the effect of the extension of homozygosity per position and per individual in relation to the effect in a reference population. This is useful in case we are considering the effect of selection in the target population while we assume no selection in the reference population. Those position that have high $iES_i$ at both populations would be considered nuisance given by other factors, like genomic effect caused by the genetic architecture of the genome. Therefore, following \citet{Tang:2007by}, we define the statistic derived from $Rsb_i$:

\begin{equation}
Rsb_{k,i} = \frac{iES_{k,i}}{iES_{i}^{popRef}}
\end{equation}
where $\sum_{k=0}^{k=n}Rsb_{k,i} = Rsb_i.$

\subsection{The study of association phenotype-genotype and the information related to the homogeneity of individuals in neighbouring regions}
To perform a study of association genotype-phenotype, we can include the information provided by the statistic $iES_{k,i}$ and $Rsb_{k,i}$ in two different ways: (a) by adding the matrix $iES$ (or $Rsbi$) in replacement of the genotype matrix or (b) using this matrix as a ponderation of the genotype matrix. %The two approaches have different meanings and thus different interpretations. In the first case, it is assumed that the homogeneity along the contiguous regions of a position  in a given individual has the same importance than the genotype of this individual, which suggest that regions close to the causal position (not necessarily causal) would be associated. In the second case, the method would stress only those genotypes with an associated phenotype that additionally would have strong ponderation (large homogeneity in the surrounding region). 

%A mixed model is used for relating the phenotype of each individual (a vector $y$ of phenotypes) with the genetic features of the individual at each position (that is, a vector $g$ containing the genotypes and the homogeneity versus surrounding positions). For each SNP we relate the phenotype using this general expression:

% \begin{equation}
%y = Xb + Zg + e ,
% \end{equation}
%where $X$ is a matrix that relates the observed with the parameters for fixed effects defined in the $b$ vector and $Z$ is a matrix that relates the observed data with the parameters related to the genotype data. $e$ is the residual error. DEFINE ASSUMPTIONS, DISTRIBUTIONS  AND SCENARIOS.

\noindent The proposal is to chose the SNPs with higher $iES$ per position (or $Rsb$ in case having a reference population) and use  these candidate positions for studying the association with the phenotype using the $iES_k$  (or $Rsb_k$) matrix.

\section{Results}
\subsection{Simulation of data under different scenarios and Validation}
\noindent -A  population suffering an strong selective sweep on a trait of interest.\par
\noindent -A  population NOT suffering an strong selective sweep on a trait of interest.\par
\noindent -Different scenarios with violations of the assumptions of a panmictic population or with violation of the effect of selection. %For example, two divergent populations under different selective forces are crossed (F1) and this one back-crossed to obtain a F2 and F3 that is genotyped and phenotyped.\par

\subsection{Real Data Analysis}
\noindent -Real data from humans?\par
\noindent -Real data from bovine?\par

\section{Discussion}


\newpage
\bibliographystyle{genetics}
\bibliography{EHHGWAS}

\end{document}
